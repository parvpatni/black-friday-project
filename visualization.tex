\documentclass[]{article}
\usepackage{lmodern}
\usepackage{amssymb,amsmath}
\usepackage{ifxetex,ifluatex}
\usepackage{fixltx2e} % provides \textsubscript
\ifnum 0\ifxetex 1\fi\ifluatex 1\fi=0 % if pdftex
  \usepackage[T1]{fontenc}
  \usepackage[utf8]{inputenc}
\else % if luatex or xelatex
  \ifxetex
    \usepackage{mathspec}
  \else
    \usepackage{fontspec}
  \fi
  \defaultfontfeatures{Ligatures=TeX,Scale=MatchLowercase}
\fi
% use upquote if available, for straight quotes in verbatim environments
\IfFileExists{upquote.sty}{\usepackage{upquote}}{}
% use microtype if available
\IfFileExists{microtype.sty}{%
\usepackage{microtype}
\UseMicrotypeSet[protrusion]{basicmath} % disable protrusion for tt fonts
}{}
\usepackage[margin=1in]{geometry}
\usepackage{hyperref}
\hypersetup{unicode=true,
            pdftitle={Black Friday Data Vosualization},
            pdfauthor={Parv},
            pdfborder={0 0 0},
            breaklinks=true}
\urlstyle{same}  % don't use monospace font for urls
\usepackage{color}
\usepackage{fancyvrb}
\newcommand{\VerbBar}{|}
\newcommand{\VERB}{\Verb[commandchars=\\\{\}]}
\DefineVerbatimEnvironment{Highlighting}{Verbatim}{commandchars=\\\{\}}
% Add ',fontsize=\small' for more characters per line
\usepackage{framed}
\definecolor{shadecolor}{RGB}{248,248,248}
\newenvironment{Shaded}{\begin{snugshade}}{\end{snugshade}}
\newcommand{\KeywordTok}[1]{\textcolor[rgb]{0.13,0.29,0.53}{\textbf{#1}}}
\newcommand{\DataTypeTok}[1]{\textcolor[rgb]{0.13,0.29,0.53}{#1}}
\newcommand{\DecValTok}[1]{\textcolor[rgb]{0.00,0.00,0.81}{#1}}
\newcommand{\BaseNTok}[1]{\textcolor[rgb]{0.00,0.00,0.81}{#1}}
\newcommand{\FloatTok}[1]{\textcolor[rgb]{0.00,0.00,0.81}{#1}}
\newcommand{\ConstantTok}[1]{\textcolor[rgb]{0.00,0.00,0.00}{#1}}
\newcommand{\CharTok}[1]{\textcolor[rgb]{0.31,0.60,0.02}{#1}}
\newcommand{\SpecialCharTok}[1]{\textcolor[rgb]{0.00,0.00,0.00}{#1}}
\newcommand{\StringTok}[1]{\textcolor[rgb]{0.31,0.60,0.02}{#1}}
\newcommand{\VerbatimStringTok}[1]{\textcolor[rgb]{0.31,0.60,0.02}{#1}}
\newcommand{\SpecialStringTok}[1]{\textcolor[rgb]{0.31,0.60,0.02}{#1}}
\newcommand{\ImportTok}[1]{#1}
\newcommand{\CommentTok}[1]{\textcolor[rgb]{0.56,0.35,0.01}{\textit{#1}}}
\newcommand{\DocumentationTok}[1]{\textcolor[rgb]{0.56,0.35,0.01}{\textbf{\textit{#1}}}}
\newcommand{\AnnotationTok}[1]{\textcolor[rgb]{0.56,0.35,0.01}{\textbf{\textit{#1}}}}
\newcommand{\CommentVarTok}[1]{\textcolor[rgb]{0.56,0.35,0.01}{\textbf{\textit{#1}}}}
\newcommand{\OtherTok}[1]{\textcolor[rgb]{0.56,0.35,0.01}{#1}}
\newcommand{\FunctionTok}[1]{\textcolor[rgb]{0.00,0.00,0.00}{#1}}
\newcommand{\VariableTok}[1]{\textcolor[rgb]{0.00,0.00,0.00}{#1}}
\newcommand{\ControlFlowTok}[1]{\textcolor[rgb]{0.13,0.29,0.53}{\textbf{#1}}}
\newcommand{\OperatorTok}[1]{\textcolor[rgb]{0.81,0.36,0.00}{\textbf{#1}}}
\newcommand{\BuiltInTok}[1]{#1}
\newcommand{\ExtensionTok}[1]{#1}
\newcommand{\PreprocessorTok}[1]{\textcolor[rgb]{0.56,0.35,0.01}{\textit{#1}}}
\newcommand{\AttributeTok}[1]{\textcolor[rgb]{0.77,0.63,0.00}{#1}}
\newcommand{\RegionMarkerTok}[1]{#1}
\newcommand{\InformationTok}[1]{\textcolor[rgb]{0.56,0.35,0.01}{\textbf{\textit{#1}}}}
\newcommand{\WarningTok}[1]{\textcolor[rgb]{0.56,0.35,0.01}{\textbf{\textit{#1}}}}
\newcommand{\AlertTok}[1]{\textcolor[rgb]{0.94,0.16,0.16}{#1}}
\newcommand{\ErrorTok}[1]{\textcolor[rgb]{0.64,0.00,0.00}{\textbf{#1}}}
\newcommand{\NormalTok}[1]{#1}
\usepackage{graphicx,grffile}
\makeatletter
\def\maxwidth{\ifdim\Gin@nat@width>\linewidth\linewidth\else\Gin@nat@width\fi}
\def\maxheight{\ifdim\Gin@nat@height>\textheight\textheight\else\Gin@nat@height\fi}
\makeatother
% Scale images if necessary, so that they will not overflow the page
% margins by default, and it is still possible to overwrite the defaults
% using explicit options in \includegraphics[width, height, ...]{}
\setkeys{Gin}{width=\maxwidth,height=\maxheight,keepaspectratio}
\IfFileExists{parskip.sty}{%
\usepackage{parskip}
}{% else
\setlength{\parindent}{0pt}
\setlength{\parskip}{6pt plus 2pt minus 1pt}
}
\setlength{\emergencystretch}{3em}  % prevent overfull lines
\providecommand{\tightlist}{%
  \setlength{\itemsep}{0pt}\setlength{\parskip}{0pt}}
\setcounter{secnumdepth}{0}
% Redefines (sub)paragraphs to behave more like sections
\ifx\paragraph\undefined\else
\let\oldparagraph\paragraph
\renewcommand{\paragraph}[1]{\oldparagraph{#1}\mbox{}}
\fi
\ifx\subparagraph\undefined\else
\let\oldsubparagraph\subparagraph
\renewcommand{\subparagraph}[1]{\oldsubparagraph{#1}\mbox{}}
\fi

%%% Use protect on footnotes to avoid problems with footnotes in titles
\let\rmarkdownfootnote\footnote%
\def\footnote{\protect\rmarkdownfootnote}

%%% Change title format to be more compact
\usepackage{titling}

% Create subtitle command for use in maketitle
\newcommand{\subtitle}[1]{
  \posttitle{
    \begin{center}\large#1\end{center}
    }
}

\setlength{\droptitle}{-2em}
  \title{Black Friday Data Vosualization}
  \pretitle{\vspace{\droptitle}\centering\huge}
  \posttitle{\par}
  \author{Parv}
  \preauthor{\centering\large\emph}
  \postauthor{\par}
  \predate{\centering\large\emph}
  \postdate{\par}
  \date{8 March 2018}


\begin{document}
\maketitle

\subsection{1. Problem Statement}\label{problem-statement}

To understand the customer purchase behaviour (specifically, purchase
amount) against various products of different categories so that sales
of a retail company can be boosted.

We will be building models to predict the purchase amount of customer
against various products which will help us to create personalized offer
for customers against different products.

\subsection{2. Setting up the
environment}\label{setting-up-the-environment}

Setting up environment for analysis, loading data, packages,
understanding variables.

\subsection{2.1 Loading Data}\label{loading-data}

\begin{Shaded}
\begin{Highlighting}[]
\KeywordTok{getwd}\NormalTok{()}
\end{Highlighting}
\end{Shaded}

\begin{verbatim}
## [1] "/home/parv/black friday project"
\end{verbatim}

\begin{Shaded}
\begin{Highlighting}[]
\NormalTok{data_set<-}\KeywordTok{read.csv}\NormalTok{(}\KeywordTok{paste}\NormalTok{(}\StringTok{"train.csv"}\NormalTok{,}\DataTypeTok{sep =} \StringTok{""}\NormalTok{))}

\NormalTok{## 75% of the sample size}
\NormalTok{smp_size <-}\StringTok{ }\KeywordTok{floor}\NormalTok{(}\FloatTok{0.75} \OperatorTok{*}\StringTok{ }\KeywordTok{nrow}\NormalTok{(data_set))}

\NormalTok{## set the seed to make your partition reproductible}
\KeywordTok{set.seed}\NormalTok{(}\DecValTok{123}\NormalTok{)}
\NormalTok{train_ind <-}\StringTok{ }\KeywordTok{sample}\NormalTok{(}\KeywordTok{seq_len}\NormalTok{(}\KeywordTok{nrow}\NormalTok{(data_set)), }\DataTypeTok{size =}\NormalTok{ smp_size)}

\NormalTok{train_set <-}\StringTok{ }\NormalTok{data_set[train_ind, ]}
\NormalTok{test_set <-}\StringTok{ }\NormalTok{data_set[}\OperatorTok{-}\NormalTok{train_ind, ]}

\NormalTok{y_test<-}\KeywordTok{as.data.frame}\NormalTok{(test_set[,}\DecValTok{12}\NormalTok{], }\DataTypeTok{drop=}\NormalTok{false)}
\KeywordTok{names}\NormalTok{(y_test)<-}\KeywordTok{c}\NormalTok{(}\StringTok{"purchase"}\NormalTok{)}

\NormalTok{test_set<-}\KeywordTok{as.data.frame}\NormalTok{(test_set[,}\DecValTok{1}\OperatorTok{:}\DecValTok{11}\NormalTok{], }\DataTypeTok{drop=}\NormalTok{false)}
\KeywordTok{dim}\NormalTok{(test_set)}
\end{Highlighting}
\end{Shaded}

\begin{verbatim}
## [1] 137517     11
\end{verbatim}

\subsection{2.2 Checking dimensions of the
data}\label{checking-dimensions-of-the-data}

\begin{Shaded}
\begin{Highlighting}[]
\KeywordTok{dim}\NormalTok{(train_set)}
\end{Highlighting}
\end{Shaded}

\begin{verbatim}
## [1] 412551     12
\end{verbatim}

\begin{Shaded}
\begin{Highlighting}[]
\KeywordTok{dim}\NormalTok{(test_set)}
\end{Highlighting}
\end{Shaded}

\begin{verbatim}
## [1] 137517     11
\end{verbatim}

\textbf{Training data} set consists of 4,12,551 entries across 12
variables.\\
\textbf{Test Data} set consists of 1,37,517 entries across 11 variables.

12th variable in Training Data set is our response Variable which needs
to be predicted in test data set.

\section{2.3 Understanding the Data
set}\label{understanding-the-data-set}

\begin{Shaded}
\begin{Highlighting}[]
\KeywordTok{library}\NormalTok{(psych)}
\KeywordTok{describe}\NormalTok{(train_set)}
\end{Highlighting}
\end{Shaded}

\begin{verbatim}
##                             vars      n       mean      sd  median
## User_ID                        1 412551 1003030.35 1728.22 1003080
## Product_ID*                    2 412551    1708.53 1011.98    1668
## Gender*                        3 412551       1.75    0.43       2
## Age*                           4 412551       3.50    1.35       3
## Occupation                     5 412551       8.08    6.53       7
## City_Category*                 6 412551       2.04    0.76       2
## Stay_In_Current_City_Years*    7 412551       2.86    1.29       3
## Marital_Status                 8 412551       0.41    0.49       0
## Product_Category_1             9 412551       5.41    3.94       5
## Product_Category_2            10 282116       9.85    5.09       9
## Product_Category_3            11 124786      12.67    4.12      14
## Purchase                      12 412551    9261.00 5022.12    8044
##                                trimmed     mad     min     max range  skew
## User_ID                     1003028.59 2180.90 1000001 1006040  6039  0.00
## Product_ID*                    1688.75 1196.46       1    3631  3630  0.15
## Gender*                           1.82    0.00       1       2     1 -1.17
## Age*                              3.36    1.48       1       7     6  0.81
## Occupation                        7.69    8.90       0      20    20  0.40
## City_Category*                    2.05    1.48       1       3     2 -0.07
## Stay_In_Current_City_Years*       2.82    1.48       1       5     4  0.32
## Marital_Status                    0.39    0.00       0       1     1  0.37
## Product_Category_1                4.91    4.45       1      20    19  1.02
## Product_Category_2               10.00    7.41       2      18    16 -0.17
## Product_Category_3               13.07    2.97       3      18    15 -0.77
## Purchase                       8925.40 4250.61      12   23961 23949  0.60
##                             kurtosis   se
## User_ID                        -1.20 2.69
## Product_ID*                    -1.09 1.58
## Gender*                        -0.63 0.00
## Age*                            0.30 0.00
## Occupation                     -1.22 0.01
## City_Category*                 -1.27 0.00
## Stay_In_Current_City_Years*    -1.07 0.00
## Marital_Status                 -1.86 0.00
## Product_Category_1              1.22 0.01
## Product_Category_2             -1.43 0.01
## Product_Category_3             -0.81 0.01
## Purchase                       -0.34 7.82
\end{verbatim}

\begin{Shaded}
\begin{Highlighting}[]
\KeywordTok{describe}\NormalTok{(test_set)}
\end{Highlighting}
\end{Shaded}

\begin{verbatim}
##                             vars      n       mean      sd  median
## User_ID                        1 137517 1003024.33 1725.69 1003070
## Product_ID*                    2 137517    1708.31 1012.87    1665
## Gender*                        3 137517       1.75    0.43       2
## Age*                           4 137517       3.49    1.36       3
## Occupation                     5 137517       8.07    6.51       7
## City_Category*                 6 137517       2.04    0.76       2
## Stay_In_Current_City_Years*    7 137517       2.86    1.29       3
## Marital_Status                 8 137517       0.41    0.49       0
## Product_Category_1             9 137517       5.38    3.93       5
## Product_Category_2            10  94314       9.82    5.08       9
## Product_Category_3            11  42035      12.67    4.13      14
##                                trimmed     mad     min     max range  skew
## User_ID                     1003021.88 2172.01 1000001 1006040  6039  0.00
## Product_ID*                    1688.48 1200.91       1    3631  3630  0.15
## Gender*                           1.82    0.00       1       2     1 -1.18
## Age*                              3.35    1.48       1       7     6  0.81
## Occupation                        7.69    8.90       0      20    20  0.40
## City_Category*                    2.05    1.48       1       3     2 -0.07
## Stay_In_Current_City_Years*       2.82    1.48       1       5     4  0.32
## Marital_Status                    0.39    0.00       0       1     1  0.37
## Product_Category_1                4.88    4.45       1      20    19  1.04
## Product_Category_2                9.97    7.41       2      18    16 -0.15
## Product_Category_3               13.07    2.97       3      18    15 -0.76
##                             kurtosis   se
## User_ID                        -1.19 4.65
## Product_ID*                    -1.09 2.73
## Gender*                        -0.60 0.00
## Age*                            0.30 0.00
## Occupation                     -1.22 0.02
## City_Category*                 -1.27 0.00
## Stay_In_Current_City_Years*    -1.07 0.00
## Marital_Status                 -1.87 0.00
## Product_Category_1              1.27 0.01
## Product_Category_2             -1.43 0.02
## Product_Category_3             -0.81 0.02
\end{verbatim}

\section{3. Univariate Analysis}\label{univariate-analysis}

Let's First Find categorical and continuos variables in our training
data set.

\begin{Shaded}
\begin{Highlighting}[]
\KeywordTok{str}\NormalTok{(train_set)}
\end{Highlighting}
\end{Shaded}

\begin{verbatim}
## 'data.frame':    412551 obs. of  12 variables:
##  $ User_ID                   : int  1000442 1000778 1004647 1002875 1001675 1003848 1002801 1003626 1004682 1002774 ...
##  $ Product_ID                : Factor w/ 3631 levels "P00000142","P00000242",..: 356 1673 2771 1017 2244 971 3510 1866 2134 1183 ...
##  $ Gender                    : Factor w/ 2 levels "F","M": 2 2 2 2 1 2 1 2 2 2 ...
##  $ Age                       : Factor w/ 7 levels "0-17","18-25",..: 3 2 4 3 3 3 3 3 3 3 ...
##  $ Occupation                : int  1 17 20 2 6 15 3 17 7 1 ...
##  $ City_Category             : Factor w/ 3 levels "A","B","C": 1 3 2 2 2 3 1 2 2 1 ...
##  $ Stay_In_Current_City_Years: Factor w/ 5 levels "0","1","2","3",..: 3 5 2 4 2 2 2 4 4 2 ...
##  $ Marital_Status            : int  0 0 1 1 1 0 1 0 1 0 ...
##  $ Product_Category_1        : int  5 8 8 1 11 5 5 5 8 1 ...
##  $ Product_Category_2        : int  NA NA NA 2 NA NA 8 11 NA 2 ...
##  $ Product_Category_3        : int  NA NA NA 8 NA NA NA NA NA 15 ...
##  $ Purchase                  : int  5472 9996 5864 19483 4592 8724 5183 5141 8050 15199 ...
\end{verbatim}

\subsection{Data Prepartion \& Exploratory Data
Analysis}\label{data-prepartion-exploratory-data-analysis}

Firstly doing the univariate exploration and modifying the data if
deemed necessary.

\textbf{one way contigency table}

\begin{Shaded}
\begin{Highlighting}[]
\NormalTok{mytable <-}\StringTok{ }\KeywordTok{with}\NormalTok{(train_set,}\KeywordTok{table}\NormalTok{(Gender))}
\NormalTok{mytable}
\end{Highlighting}
\end{Shaded}

\begin{verbatim}
## Gender
##      F      M 
## 102021 310530
\end{verbatim}

\begin{Shaded}
\begin{Highlighting}[]
\NormalTok{lbls <-}\StringTok{ }\KeywordTok{c}\NormalTok{(}\StringTok{"M"}\NormalTok{,}\StringTok{"F"}\NormalTok{)}
\NormalTok{pct <-}\StringTok{ }\KeywordTok{round}\NormalTok{(mytable}\OperatorTok{/}\KeywordTok{sum}\NormalTok{(mytable)}\OperatorTok{*}\DecValTok{100}\NormalTok{)}
\NormalTok{lbls <-}\StringTok{ }\KeywordTok{paste}\NormalTok{(lbls, pct) }
\NormalTok{lbls <-}\StringTok{ }\KeywordTok{paste}\NormalTok{(lbls,}\StringTok{"%"}\NormalTok{,}\DataTypeTok{sep=}\StringTok{""}\NormalTok{)}
\KeywordTok{pie}\NormalTok{(mytable,}\DataTypeTok{labels =}\NormalTok{ lbls)}
\end{Highlighting}
\end{Shaded}

\includegraphics{visualization_files/figure-latex/unnamed-chunk-5-1.pdf}

\begin{Shaded}
\begin{Highlighting}[]
\NormalTok{mytable1 <-}\StringTok{ }\KeywordTok{with}\NormalTok{(train_set,}\KeywordTok{table}\NormalTok{(Marital_Status))}
\NormalTok{mytable1}
\end{Highlighting}
\end{Shaded}

\begin{verbatim}
## Marital_Status
##      0      1 
## 243555 168996
\end{verbatim}

\begin{Shaded}
\begin{Highlighting}[]
\NormalTok{lbls1 <-}\StringTok{ }\KeywordTok{c}\NormalTok{(}\StringTok{"Single"}\NormalTok{,}\StringTok{"Married"}\NormalTok{)}
\NormalTok{pct1 <-}\StringTok{ }\KeywordTok{round}\NormalTok{(mytable1}\OperatorTok{/}\KeywordTok{sum}\NormalTok{(mytable1)}\OperatorTok{*}\DecValTok{100}\NormalTok{)}
\NormalTok{lbls1 <-}\StringTok{ }\KeywordTok{paste}\NormalTok{(lbls1, pct1) }
\NormalTok{lbls1<-}\StringTok{ }\KeywordTok{paste}\NormalTok{(lbls1,}\StringTok{"%"}\NormalTok{,}\DataTypeTok{sep=}\StringTok{""}\NormalTok{)}
\KeywordTok{pie}\NormalTok{(mytable1,}\DataTypeTok{labels =}\NormalTok{ lbls1)}
\end{Highlighting}
\end{Shaded}

\includegraphics{visualization_files/figure-latex/unnamed-chunk-6-1.pdf}

\begin{Shaded}
\begin{Highlighting}[]
\NormalTok{mytable2 <-}\StringTok{ }\KeywordTok{with}\NormalTok{(train_set,}\KeywordTok{table}\NormalTok{(City_Category))}
\NormalTok{mytable2}
\end{Highlighting}
\end{Shaded}

\begin{verbatim}
## City_Category
##      A      B      C 
## 110815 173429 128307
\end{verbatim}

\begin{Shaded}
\begin{Highlighting}[]
\NormalTok{lbls2 <-}\StringTok{ }\KeywordTok{c}\NormalTok{(}\StringTok{"A"}\NormalTok{,}\StringTok{"B"}\NormalTok{,}\StringTok{"C"}\NormalTok{)}
\NormalTok{pct2 <-}\StringTok{ }\KeywordTok{round}\NormalTok{(mytable2}\OperatorTok{/}\KeywordTok{sum}\NormalTok{(mytable2)}\OperatorTok{*}\DecValTok{100}\NormalTok{)}
\NormalTok{lbls2 <-}\StringTok{ }\KeywordTok{paste}\NormalTok{(lbls2, pct2) }
\NormalTok{lbls2<-}\StringTok{ }\KeywordTok{paste}\NormalTok{(lbls2,}\StringTok{"%"}\NormalTok{,}\DataTypeTok{sep=}\StringTok{""}\NormalTok{)}
\KeywordTok{pie}\NormalTok{(mytable2,}\DataTypeTok{labels =}\NormalTok{ lbls2)}
\end{Highlighting}
\end{Shaded}

\includegraphics{visualization_files/figure-latex/unnamed-chunk-7-1.pdf}

\begin{Shaded}
\begin{Highlighting}[]
\KeywordTok{par}\NormalTok{(}\DataTypeTok{mfrow =} \KeywordTok{c}\NormalTok{(}\DecValTok{1}\NormalTok{, }\DecValTok{2}\NormalTok{))}
\CommentTok{#train data}
\KeywordTok{with}\NormalTok{(train_set,}\KeywordTok{table}\NormalTok{(Stay_In_Current_City_Years))}
\end{Highlighting}
\end{Shaded}

\begin{verbatim}
## Stay_In_Current_City_Years
##      0      1      2      3     4+ 
##  55676 145504  76472  71518  63381
\end{verbatim}

\begin{Shaded}
\begin{Highlighting}[]
\CommentTok{#test data}
\KeywordTok{with}\NormalTok{(test_set,}\KeywordTok{table}\NormalTok{(Stay_In_Current_City_Years))}
\end{Highlighting}
\end{Shaded}

\begin{verbatim}
## Stay_In_Current_City_Years
##     0     1     2     3    4+ 
## 18722 48317 25366 23767 21345
\end{verbatim}

\begin{Shaded}
\begin{Highlighting}[]
\KeywordTok{par}\NormalTok{(}\DataTypeTok{mfrow =} \KeywordTok{c}\NormalTok{(}\DecValTok{1}\NormalTok{, }\DecValTok{2}\NormalTok{))}
\NormalTok{a<-}\KeywordTok{with}\NormalTok{(train_set,}\KeywordTok{table}\NormalTok{(Product_Category_}\DecValTok{1}\NormalTok{))}
\CommentTok{#a}
\KeywordTok{barplot}\NormalTok{(a,}\DataTypeTok{main =}\StringTok{"train_data"}\NormalTok{)}
\NormalTok{b<-}\KeywordTok{with}\NormalTok{(test_set,}\KeywordTok{table}\NormalTok{(Product_Category_}\DecValTok{1}\NormalTok{))}
\CommentTok{#b}
\KeywordTok{barplot}\NormalTok{(b,}\DataTypeTok{main =}\StringTok{"test_data"}\NormalTok{)}
\end{Highlighting}
\end{Shaded}

\includegraphics{visualization_files/figure-latex/unnamed-chunk-9-1.pdf}

\begin{Shaded}
\begin{Highlighting}[]
\KeywordTok{par}\NormalTok{(}\DataTypeTok{mfrow =} \KeywordTok{c}\NormalTok{(}\DecValTok{1}\NormalTok{, }\DecValTok{2}\NormalTok{))}
\NormalTok{a<-}\KeywordTok{with}\NormalTok{(train_set,}\KeywordTok{table}\NormalTok{(Product_Category_}\DecValTok{2}\NormalTok{))}
\CommentTok{#a}
\KeywordTok{barplot}\NormalTok{(a,}\DataTypeTok{main =}\StringTok{"train_data"}\NormalTok{)}
\NormalTok{b<-}\KeywordTok{with}\NormalTok{(test_set,}\KeywordTok{table}\NormalTok{(Product_Category_}\DecValTok{2}\NormalTok{))}
\CommentTok{#b}
\KeywordTok{barplot}\NormalTok{(b,}\DataTypeTok{main =}\StringTok{"test_data"}\NormalTok{)}
\end{Highlighting}
\end{Shaded}

\includegraphics{visualization_files/figure-latex/unnamed-chunk-10-1.pdf}

\begin{Shaded}
\begin{Highlighting}[]
\KeywordTok{library}\NormalTok{(lattice)}
\KeywordTok{par}\NormalTok{(}\DataTypeTok{mfrow =} \KeywordTok{c}\NormalTok{(}\DecValTok{1}\NormalTok{, }\DecValTok{2}\NormalTok{))}
\NormalTok{a<-}\KeywordTok{with}\NormalTok{(train_set,}\KeywordTok{table}\NormalTok{(Product_Category_}\DecValTok{3}\NormalTok{))}
\CommentTok{#a}
\KeywordTok{barchart}\NormalTok{(a,}\DataTypeTok{main =}\StringTok{"train_data"}\NormalTok{,}\DataTypeTok{horizontal =} \StringTok{"FALSE"}\NormalTok{)}
\end{Highlighting}
\end{Shaded}

\includegraphics{visualization_files/figure-latex/unnamed-chunk-11-1.pdf}

\begin{Shaded}
\begin{Highlighting}[]
\NormalTok{b<-}\KeywordTok{with}\NormalTok{(test_set,}\KeywordTok{table}\NormalTok{(Product_Category_}\DecValTok{3}\NormalTok{))}
\CommentTok{#b}
\KeywordTok{barchart}\NormalTok{(b,}\DataTypeTok{main =}\StringTok{"test_data"}\NormalTok{,}\DataTypeTok{horizontal =} \StringTok{"FALSE"}\NormalTok{)}
\end{Highlighting}
\end{Shaded}

\includegraphics{visualization_files/figure-latex/unnamed-chunk-11-2.pdf}

\textbf{2 way table}

\begin{Shaded}
\begin{Highlighting}[]
\KeywordTok{xtabs}\NormalTok{(}\OperatorTok{~}\NormalTok{Marital_Status}\OperatorTok{+}\NormalTok{Gender,}\DataTypeTok{data=}\NormalTok{train_set)}
\end{Highlighting}
\end{Shaded}

\begin{verbatim}
##               Gender
## Marital_Status      F      M
##              0  59207 184348
##              1  42814 126182
\end{verbatim}

\begin{Shaded}
\begin{Highlighting}[]
\KeywordTok{xtabs}\NormalTok{(Purchase}\OperatorTok{~}\NormalTok{Product_Category_}\DecValTok{3}\NormalTok{,}\KeywordTok{aggregate}\NormalTok{(Purchase}\OperatorTok{~}\NormalTok{Product_Category_}\DecValTok{3}\NormalTok{,train_set,mean))}
\end{Highlighting}
\end{Shaded}

\begin{verbatim}
## Product_Category_3
##         3         4         5         6         8         9        10 
## 14058.795  9806.369 12128.368 13179.604 13047.715 10388.366 13507.104 
##        11        12        13        14        15        16        17 
## 12113.981  8703.027 13237.030 10036.091 12342.484 11960.804 11737.853 
##        18 
## 11015.255
\end{verbatim}

\begin{Shaded}
\begin{Highlighting}[]
\KeywordTok{xtabs}\NormalTok{(Purchase}\OperatorTok{~}\NormalTok{Product_Category_}\DecValTok{2}\NormalTok{,}\KeywordTok{aggregate}\NormalTok{(Purchase}\OperatorTok{~}\NormalTok{Product_Category_}\DecValTok{2}\NormalTok{,train_set,mean))}
\end{Highlighting}
\end{Shaded}

\begin{verbatim}
## Product_Category_2
##         2         3         4         5         6         7         8 
## 13610.833 11222.781 10216.105  9033.057 11516.350  6894.512 10247.663 
##         9        10        11        12        13        14        15 
##  7329.170 15715.654  8893.459  6996.297  9662.622  7101.230 10345.805 
##        16        17        18 
## 10289.805  9421.296  9328.248
\end{verbatim}

\begin{Shaded}
\begin{Highlighting}[]
\KeywordTok{xtabs}\NormalTok{(Purchase}\OperatorTok{~}\NormalTok{Product_Category_}\DecValTok{1}\NormalTok{,}\KeywordTok{aggregate}\NormalTok{(Purchase}\OperatorTok{~}\NormalTok{Product_Category_}\DecValTok{1}\NormalTok{,train_set,mean))}
\end{Highlighting}
\end{Shaded}

\begin{verbatim}
## Product_Category_1
##           1           2           3           4           5           6 
## 13600.94860 11232.00870 10109.66465  2334.29817  6240.79881 15830.96146 
##           7           8           9          10          11          12 
## 16374.65185  7497.88064 15754.68932 19668.12699  4687.24631  1344.33266 
##          13          14          15          16          17          18 
##   720.56800 13143.73906 14790.16472 14754.58678 10167.59811  2979.82756 
##          19          20 
##    36.78061   369.85307
\end{verbatim}

\subsection{3.1. Unique Data For EDA \& Data type
modification}\label{unique-data-for-eda-data-type-modification}

Since our data is stored based on product id i.e.~if a person buying 10
products then his data will be stored in 10 observation and hence there
will cause a repetition of same person data.

\begin{Shaded}
\begin{Highlighting}[]
\NormalTok{train_set}\OperatorTok{$}\NormalTok{User_ID <-}\StringTok{ }\KeywordTok{as.factor}\NormalTok{(train_set}\OperatorTok{$}\NormalTok{User_ID)}
\NormalTok{train_set}\OperatorTok{$}\NormalTok{Product_ID <-}\StringTok{ }\KeywordTok{as.factor}\NormalTok{(train_set}\OperatorTok{$}\NormalTok{Product_ID)}
\NormalTok{train_set}\OperatorTok{$}\NormalTok{Marital_Status <-}\StringTok{ }\KeywordTok{as.factor}\NormalTok{(}\KeywordTok{ifelse}\NormalTok{(train_set}\OperatorTok{$}\NormalTok{Marital_Status }\OperatorTok{==}\StringTok{ }\DecValTok{1}\NormalTok{, }\StringTok{'Married'}\NormalTok{, }\StringTok{'Single'}\NormalTok{))}
\NormalTok{train_set}\OperatorTok{$}\NormalTok{Age <-}\StringTok{ }\KeywordTok{as.factor}\NormalTok{(train_set}\OperatorTok{$}\NormalTok{Age)}
\NormalTok{train_set}\OperatorTok{$}\NormalTok{Gender <-}\StringTok{ }\KeywordTok{as.factor}\NormalTok{(}\KeywordTok{ifelse}\NormalTok{(train_set}\OperatorTok{$}\NormalTok{Gender}\OperatorTok{==}\StringTok{'M'}\NormalTok{, }\StringTok{'Male'}\NormalTok{, }\StringTok{'Female'}\NormalTok{))}
\NormalTok{train_set}\OperatorTok{$}\NormalTok{Occupation <-}\StringTok{ }\KeywordTok{as.factor}\NormalTok{(train_set}\OperatorTok{$}\NormalTok{Occupation)}
\NormalTok{train_set}\OperatorTok{$}\NormalTok{City_Category <-}\StringTok{ }\KeywordTok{as.factor}\NormalTok{(train_set}\OperatorTok{$}\NormalTok{City_Category)}
\NormalTok{train_set}\OperatorTok{$}\NormalTok{Stay_In_Current_City_Years <-}\StringTok{ }\KeywordTok{as.factor}\NormalTok{(train_set}\OperatorTok{$}\NormalTok{Stay_In_Current_City_Years)}

\NormalTok{test_set}\OperatorTok{$}\NormalTok{User_ID <-}\StringTok{ }\KeywordTok{as.factor}\NormalTok{(test_set}\OperatorTok{$}\NormalTok{User_ID)}
\NormalTok{test_set}\OperatorTok{$}\NormalTok{Product_ID <-}\StringTok{ }\KeywordTok{as.factor}\NormalTok{(test_set}\OperatorTok{$}\NormalTok{Product_ID)}
\NormalTok{test_set}\OperatorTok{$}\NormalTok{Marital_Status <-}\StringTok{ }\KeywordTok{as.factor}\NormalTok{(}\KeywordTok{ifelse}\NormalTok{(test_set}\OperatorTok{$}\NormalTok{Marital_Status }\OperatorTok{==}\StringTok{ }\DecValTok{1}\NormalTok{, }\StringTok{'Married'}\NormalTok{, }\StringTok{'Single'}\NormalTok{))}
\NormalTok{test_set}\OperatorTok{$}\NormalTok{Age <-}\StringTok{ }\KeywordTok{as.factor}\NormalTok{(test_set}\OperatorTok{$}\NormalTok{Age)}
\NormalTok{test_set}\OperatorTok{$}\NormalTok{Gender <-}\StringTok{ }\KeywordTok{as.factor}\NormalTok{(}\KeywordTok{ifelse}\NormalTok{(test_set}\OperatorTok{$}\NormalTok{Gender}\OperatorTok{==}\StringTok{'M'}\NormalTok{, }\StringTok{'Male'}\NormalTok{, }\StringTok{'Female'}\NormalTok{))}
\NormalTok{test_set}\OperatorTok{$}\NormalTok{Occupation <-}\StringTok{ }\KeywordTok{as.factor}\NormalTok{(test_set}\OperatorTok{$}\NormalTok{Occupation)}
\NormalTok{test_set}\OperatorTok{$}\NormalTok{City_Category <-}\StringTok{ }\KeywordTok{as.factor}\NormalTok{(test_set}\OperatorTok{$}\NormalTok{City_Category)}
\NormalTok{test_set}\OperatorTok{$}\NormalTok{Stay_In_Current_City_Years <-}\StringTok{ }\KeywordTok{as.factor}\NormalTok{(test_set}\OperatorTok{$}\NormalTok{Stay_In_Current_City_Years)}

\CommentTok{#str(train_set)}
\CommentTok{#str(test_set)}
\end{Highlighting}
\end{Shaded}

The function distinct() in dplyr package can be used to keep only
unique/distinct rows from a data frame. If there are duplicate rows,
only the first row is preserved. It's an efficient version of the R base
function unique().

\begin{Shaded}
\begin{Highlighting}[]
\KeywordTok{library}\NormalTok{(}\StringTok{"dplyr"}\NormalTok{)}
\end{Highlighting}
\end{Shaded}

\begin{verbatim}
## 
## Attaching package: 'dplyr'
\end{verbatim}

\begin{verbatim}
## The following objects are masked from 'package:stats':
## 
##     filter, lag
\end{verbatim}

\begin{verbatim}
## The following objects are masked from 'package:base':
## 
##     intersect, setdiff, setequal, union
\end{verbatim}

\begin{Shaded}
\begin{Highlighting}[]
\NormalTok{EDA_Distinct <-}\StringTok{ }\KeywordTok{distinct}\NormalTok{(train_set, User_ID, Age, Gender, Marital_Status, Occupation, City_Category, Stay_In_Current_City_Years)}
\CommentTok{#str(EDA_Distinct)}
\KeywordTok{head}\NormalTok{(EDA_Distinct)}
\end{Highlighting}
\end{Shaded}

\begin{verbatim}
##   User_ID Gender   Age Occupation City_Category Stay_In_Current_City_Years
## 1 1000442   Male 26-35          1             A                          2
## 2 1000778   Male 18-25         17             C                         4+
## 3 1004647   Male 36-45         20             B                          1
## 4 1002875   Male 26-35          2             B                          3
## 5 1001675 Female 26-35          6             B                          1
## 6 1003848   Male 26-35         15             C                          1
##   Marital_Status
## 1         Single
## 2         Single
## 3        Married
## 4        Married
## 5        Married
## 6         Single
\end{verbatim}

\section{3.2. User\_ID}\label{user_id}

**


\end{document}
